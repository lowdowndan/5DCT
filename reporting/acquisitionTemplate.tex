\documentclass[12pt]{article}

\usepackage{amsmath,amsthm,amssymb}
\usepackage{fixltx2e}
\usepackage{float}
\usepackage{booktabs}
\usepackage{siunitx}
\usepackage{caption}
\usepackage[T1]{fontenc}
\usepackage{graphicx} 
\usepackage[default]{droidserif}
\usepackage{fancyhdr}
\usepackage[margin=0.75in,headsep=0.25in]{geometry}

\pagestyle{fancy}
\bibliographystyle{plain}

% --1.

\chead{}
% --2.

\lfoot{}
\cfoot{\thepage}
\rfoot{}
\renewcommand{\headrulewidth}{0.4pt}
\addtolength{\headheight}{1.30\baselineskip}
\addtolength{\headheight}{0.61pt}


\begin{document}
\section*{5DCT Acquisition Report}


\noindent Patient was imaged multiple times in succession using a
free-breathing, fast helical protocol.  The respiratory cycle was monitored
using using a pneumatic abdominal bellows device and a pressure transducer to
generate a signal proportional to tidal volume.  Free-breathing scans were
deformably registered to a common geometry to measure displacement of tissue.
Tissue motion was correlated to breathing amplitude and rate using a linear
model. \\
% --17.

\noindent Eight respiratory-gated images were generated at breathing phases
corresponding to the reconstruction points of the Siemens Sensation Open 4DCT protocol.
% --3.

\medskip


\begin{table}[h!]
	\centering
\begin{tabular}{cc} \toprule
	\midrule
% --4.
% --5.
% --6.
% --7.
% --8.
% --9.

\bottomrule
\end{tabular}
\end{table}

\begin{figure}[h!]
	\centering
% --10.

\end{figure}
	
\clearpage
\begin{figure}[h!]
	\centering
% --11.
\end{figure}

\begin{table}[h!]
	\centering
\begin{tabular}{cc} \toprule
Amplitude Percentile Interval & Percent Time Included\\ \midrule

% --12.
% --13.
% --14.
% --15.
% --16.

\bottomrule
\end{tabular}
\end{table}
\end{document}

